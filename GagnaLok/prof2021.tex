\documentclass{article}

% Encodings, page setup, paragraph formatting, font
\usepackage[top=0.9in, bottom=1in, left=1.5in, right=1.5in]{geometry}
\usepackage[icelandic]{babel}
\usepackage[T1]{fontenc}
\usepackage[sc]{mathpazo}
\usepackage[parfill]{parskip}
\usepackage{cancel}
\usepackage{comment}
% Tables and lists
\usepackage{booktabs,tabularx}
\usepackage{multirow}
\usepackage{enumerate}
\usepackage{adjustbox}
\usepackage{multicol}
\usepackage{enumitem}
\usepackage{xcolor}
% Math
\usepackage{amsmath, amsfonts, amssymb, amsthm}
\usepackage{gensymb}
% Graphics
\usepackage{graphicx}
\usepackage{forest}
\usepackage{tikz}
\usetikzlibrary{positioning, shapes, arrows.meta}
% Code environment
\usepackage{listingsutf8}
\definecolor{commentcolor}{RGB}{0, 128, 0} % Grænn
\definecolor{keywordcolor}{RGB}{0, 0, 255}   % Blár
\definecolor{stringcolor}{RGB}{163, 21, 21}      % Dökkrauður
\definecolor{numbercolor}{RGB}{128, 0, 128}      % Fjólublár
\definecolor{identifiercolor}{RGB}{0, 0, 0}      % Svartur

\def\ojoin{\setbox0=\hbox{$\bowtie$}%
  \rule[-.02ex]{.25em}{.4pt}\llap{\rule[\ht0]{.25em}{.4pt}}}
\def\leftouterjoin{\mathbin{\ojoin\mkern-5.8mu\bowtie}}
\def\rightouterjoin{\mathbin{\bowtie\mkern-5.8mu\ojoin}}
\def\fullouterjoin{\mathbin{\ojoin\mkern-5.8mu\bowtie\mkern-5.8mu\ojoin}}


\lstset{
    language=Java,
    basicstyle=\ttfamily,
    keywordstyle=\color{keywordcolor}\bfseries,
    commentstyle=\color{commentcolor},
    identifierstyle=\color{identifiercolor},
    stringstyle=\color{stringcolor},   
    showstringspaces=false,
    numbers=left,
    numberstyle=\tiny\color{gray},
    tabsize=2,
    breaklines=true,
    columns=fullflexible,
    keepspaces=true,
    inputencoding=utf8, 
    extendedchars=true,  
    literate=
        {á}{{\'a}}1
        {ð}{{\dh}}1
        {é}{{\'e}}1
        {í}{{\'i}}1
        {ó}{{\'o}}1
        {ú}{{\'u}}1
        {ý}{{\'y}}1
        {þ}{{\th}}1
        {æ}{{\ae}}1
        {ö}{{\"o}}1
        {Á}{{\'A}}1
        {Ð}{{\DH}}1
        {É}{{\'E}}1
        {Í}{{\'I}}1
        {Ó}{{\'O}}1
        {Ú}{{\'U}}1
        {Ý}{{\'Y}}1
        {Þ}{{\TH}}1
        {Æ}{{\AE}}1
        {Ö}{{\"O}}1,
}

% Restin af forskriftinni
\usepackage[pdftex,bookmarks=true,colorlinks=true,pdfauthor={Hafsteinn Einarsson},linkcolor=blue,urlcolor=blue]{hyperref}

%Custom Commands til að auðvelda mér lífið
\newcommand{\sv}{\textbf{Svar:}}
\newcommand{\bo}[1]{\textbf{#1}}
\newcommand{\enum}{\begin{enumerate}[label = \alph*.]}

% Hyphenation
\hyphenpenalty=5000
% Page and section numbering
\setcounter{secnumdepth}{-1} 
\pagenumbering{gobble}

\title{Gagnasafnspróf Lokaprófa undirbúningur próf 2021}
\author{brj46 }
\date{Desember 2024}

\begin{document}

\maketitle

\begin{center}
\LARGE{\textbf{Mínar Lausnir á 2021 prófinu}}
\end{center}

\newpage

\begin{center}
    \bo{Hluti I - SQL o.fl.}
\end{center}

\bo{Svarið að minnsta kosti þremur spurningum í þessum hluta og a.m.k. 10 spurningum í heild}


Gerið ráð fyrir að kvikmyndagagnagrunnurinn sé skilgreindur með
eftirfarandi töfluskilgreiningum.

\begin{verbatim}
CREATE TABLE MovieExec
    ( name VARCHAR(25)
    , address VARCHAR(25)
    , cert VARCHAR(3) PRIMARY KEY
    , netWorth INT
    );
CREATE TABLE Studio
    ( name VARCHAR(25) PRIMARY KEY
    , address VARCHAR(25)
    , presC VARCHAR(3) REFERENCES MovieExec(cert)
    );
CREATE TABLE Movie
    ( title VARCHAR(25)
    , year INT
    , length INT
    , inColor BOOLEAN
    , studioName VARCHAR(25) REFERENCES Studio(name)
    , producerC VARCHAR(3) REFERENCES MovieExec(cert)
    , PRIMARY KEY(title,year)
    );
CREATE TABLE MovieStar
    ( name VARCHAR(25) PRIMARY KEY
    , address VARCHAR(25)
    , gender CHAR(1)
    , birthdate VARCHAR(10)
    );
CREATE TABLE StarsIn
    ( movieTitle VARCHAR(25)
    , movieYear INT
    , starName VARCHAR(25) REFERENCES MovieStar(name)
    , PRIMARY KEY(movieTitle,movieYear,starName)
    , FOREIGN KEY(movieTitle,movieYear)
    REFERENCES Movie(title,year)
    );

\end{verbatim}
\section{1.}
Skrifið SQL fyrirspurnir fyrir eftirfarandi. Write SQL queries for the
following.

\enum
\item Finnið titil og ár kvikmyndarinnar sem hefur titil sem er fremst
í stafrófsröð allra titla.
\item Finnið nöfn þeirra kvikmyndaframleiðenda sem ekki hafa
framleitt neina kvikmynd.
\item Finnið nöfn þeirra kvikmyndavera (studio) sem ekki hafa
framleitt kvikmynd sem Jack Nicholson lék í.
\end{enumerate}

\newpage

\section{2.}
Skrifið SQL fyrirspurnir fyrir eftirfarandi.

\enum
\item Finnið nöfn þeirra kvikmynda sem hafa samnefnda kvikmynd.
\item Finnið nöfn þeirra kvikmyndastjarna sem hafa leikið með
öllum öðrum kvenkyns kvikmyndastjörnum í einhverri kvikmynd.
\item Finnið fyrir sérhverja kvikmyndastjörnu, sem leikið hefur í
a.m.k. tveimur kvikmyndum, heildarlengd allra kvikmynda hennar
(eða hans).
\end{enumerate}

\vspace{1cm}

\section{3}
\enum
\item Finnið fyrir alla kvikmyndaframleiðendur hve margar
kvikmyndir framleiðandinn hefur framleitt og heildarlengd 
kvikmyndanna. Athugið að kvikmyndaframleiðandi er ekki sama og
kvikmyndaver. 
\item Finnið nafn og heimilisfang þess kvikmyndavers sem framleitt
hefur mestu heildarlengd kvikmynda.
\item Finnið nafn og heimilisfang þeirrar kvikmyndastjörnu sem
leikið hefur í mestri heildarlengd kvikmynda.
\end{enumerate}

\vspace{1cm}

\section{4.}

\enum 
\item Finnið þær kvikmyndastjörnur sem léku í öllum kvikmyndum
þar sem titillinn byrjar á stöfunum ‘Star Trek’.
\item Finnið nafn þeirrar kvikmyndastjörnu sem leikið hafa í
kvikmyndum með flestum framleiðanda.
\end{enumerate}

\vspace{1cm}

\section{5.}
Miðað við að C og D séu dálkar af tagi VARCHAR(30), hverjar af
eftirfarandi segðum skila alltaf TRUE?

\enum
\item C LIKE D OR C IS NULL
\item D LIKE D OR D IS NULL
\item C NOT LIKE D OR C LIKE D
\item C IS NULL OR D IS NULL OR C LIKE D OR D LIKE C
\end{enumerate}

\newpage

\section{6.}
Gerið ráð fyrir að gagnagrunnsnotandi A eigi allar töflurnar í
kvikmyndagagnagrunninum. Gerið ráð fyrir að notendur A...E
framkvæmi eftirfarandi skipanir í eftirfarandi röð.

\begin{itemize}
\item \bo{A:} GRANT SELECT ON Studio TO B,C WITH GRANT OPTION;
\item \bo{B:} GRANT SELECT ON Studio TO C,D WITH GRANT OPTION;
\item \bo{C:} GRANT SELECT ON Studio TO E;
\item \bo{B:} REVOKE SELECT ON Studio FROM C CASCADE;
\end{itemize}
Hverjar af eftirfarandi skipunum frá notendum A...E munu þá keyra
án villu?
\enum
    \item \bo{A:} SELECT * FROM Studio;
    \item \bo{B:} SELECT * FROM Studio;
    \item \bo{C:} SELECT * FROM Studio;
    \item \bo{D:} SELECT * FROM Studio; 
    \item \bo{E:} SELECT * FROM Studio;
    \item \bo{A:} REVOKE SELECT ON Studio FROM C RESTRICT;
\end{enumerate}

\newpage

\begin{center}
    \bo{Hluti II - Venslaalgebra o.fl.}

    Svarið að minnsta kosti einni spurningu í þessum hluta og a.m.k.
    10 spurningum í heild
\end{center}

\section{7.}
Skrifið venslaalgebrusegðir sem eru jafngildar eftirfarandi SQL
fyrirspurnum fyrir töflur $R(A, B, C)$ og $S(C, D).$

\enum
\item SELECT A,B FROM R,S WHERE B=D
\item SELECT A,B,D FROM R NATURAL JOIN S
\item SELECT A,B FROM R RIGHT OUTER NATURAL JOIN S
\end{enumerate}

\vspace{1cm}

\section{8}
Hvaða segðir í venslaalgebru gefa sömu útkomu og eftirfarandi
SQL fyrirspurnir fyrir töflur $R(A, B, C)$ og $S(B, C, D)$? Tiltakið núll eða
fleiri venslaalgebrusegðir fyrir hverja SQL fyrirspurn.

\bo{SQL fyrirspurnir}
\begin{itemize}
    \item A. SELECT A FROM R LEFT OUTER NATURAL JOIN S
    \item B. SELECT A FROM R,S WHERE R.B=S.B AND R.C=S.C
    \item C. SELECT A FROM R NATURAL JOIN S
\end{itemize}
\bo{Venslaalgerbusegðir}
\enum
\item $\pi_A(\sigma_{R.B=S.B\wedge R.C=S.C}(R\times S))$
\item $\pi_A(R \bowtie S)$
\item $\pi_A(R \leftouterjoin_{B=C}S)$
\item $\pi_A(R \bowtie_{B=C}S)$
\item $\pi_A(R \leftouterjoin S)$
\end{enumerate}

\end{document}